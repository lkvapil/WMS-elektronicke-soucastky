\chapter{Popis procesů}
\label{chap:procesy}

\section{Proces příjmu součástek}

Příjem součástek na sklad je jedním z~nejčastějších procesů. Probíhá typicky
na základě dodávky od dodavatele, přičemž může, ale nemusí, navazovat na
předchozí objednávku.

Sekvenční diagram příjmu je zobrazen na \autoref{fig:seq_prijem}.

\begin{figure}[H]
  \centering
  \includegraphics[width=0.9\textwidth]{04_sequence_prijem.png}
  \caption{Sekvenční diagram -- Příjem součástek na sklad}
  \label{fig:seq_prijem}
\end{figure}

\subsection{Popis kroků příjmu}

\begin{enumerate}
  \item Skladník otevře formulář příjmu a naskenuje/zadá číslo dokladu nebo
        vyhledá objednávku.
  \item Pro každou přijímanou položku:
    \begin{enumerate}
      \item Vyhledá součástku (dle part number nebo skenování čárového kódu).
      \item Určí cílové skladové místo.
      \item Zadá přijímané množství a jednotkovou cenu.
    \end{enumerate}
  \item Systém ověří validitu dat (množství $> 0$, místo existuje, součástka aktivní).
  \item Systém vytvoří záznamy v~\texttt{stock\_movements} a aktualizuje tabulku
        \texttt{stock}.
  \item Pokud příjem navazuje na objednávku, systém aktualizuje stav objednávky.
  \item Systém zkontroluje, zda příjem neobnovil zásoby nad minimum (zruší upozornění).
  \item Vytiskne se příjemka.
\end{enumerate}

\section{Proces výdeje součástek}

Výdej se provádí při přidělení součástek konkrétnímu projektu nebo opravě.
Sekvenční diagram je na \autoref{fig:seq_vydej}.

\begin{figure}[H]
  \centering
  \includegraphics[width=0.9\textwidth]{05_sequence_vydej.png}
  \caption{Sekvenční diagram -- Výdej součástek ze skladu}
  \label{fig:seq_vydej}
\end{figure}

\section{Proces inventury}

Inventura je periodický proces ověření fyzického stavu skladu. Diagram
aktivit je zobrazen na \autoref{fig:activity_inventura}.

\begin{figure}[H]
  \centering
  \includegraphics[width=0.85\textwidth]{06_activity_inventura.png}
  \caption{Diagram aktivit -- Průběh inventury}
  \label{fig:activity_inventura}
\end{figure}

\subsection{Fáze inventurního procesu}

\begin{description}
  \item[Příprava] Administrátor zahájí inventurní cyklus. Systém uzamkne zásoby
        pro editaci a vygeneruje inventurní listy dle skladových míst.
  \item[Fyzické sčítání] Skladník prochází místa a zadává fyzicky zjištěné
        množství. Systém umožňuje zadávání postupně po sekcích.
  \item[Porovnání] Systém porovná evidované a fyzicky zjištěné stavy.
        Zobrazí seznam rozdílů (přebytky / manka).
  \item[Schválení a korekce] Administrátor přezkoumá rozdíly. Po schválení
        systém vytvoří pohyby typu \texttt{ADJUSTMENT} a aktualizuje zásoby.
  \item[Uzavření] Inventura je uzavřena, zásoby odemčeny, vygenerován protokol.
\end{description}

\section{Stav objednávky -- stavový diagram}

\begin{figure}[H]
  \centering
  % Jednoduchý stavový diagram pomocí TikZ (bez PlantUML obrázku)
  \begin{tikzpicture}[
    state/.style={draw, rounded corners, minimum width=2.5cm, minimum height=0.8cm,
                  fill=blue!10, font=\small},
    arrow/.style={-{Latex[length=2mm]}, thick}
  ]
    \node[state] (draft)     at (0,0)    {NÁVRH};
    \node[state] (sent)      at (4,0)    {ODESLÁNO};
    \node[state] (confirmed) at (8,0)    {POTVRZENO};
    \node[state] (received)  at (8,-2)   {PŘIJATO};
    \node[state] (cancelled) at (4,-2)   {STORNOVÁNO};

    \draw[arrow] (draft)     -- node[above,font=\tiny]{odeslat} (sent);
    \draw[arrow] (sent)      -- node[above,font=\tiny]{potvrdit} (confirmed);
    \draw[arrow] (confirmed) -- node[right,font=\tiny]{přijmout} (received);
    \draw[arrow] (sent)      -- node[right,font=\tiny]{storno} (cancelled);
    \draw[arrow] (confirmed) -- node[above,font=\tiny]{storno} (cancelled);
    \draw[arrow] (draft)     to[bend right=30] node[below,font=\tiny]{storno} (cancelled);
  \end{tikzpicture}
  \caption{Stavový diagram objednávky}
  \label{fig:stav_objednavky}
\end{figure}
