\chapter{Závěr}
\label{chap:zaver}

\section{Shrnutí}

Tato dokumentace popsala návrh systému pro správu skladu elektronických
součástek (\wms{}). Systém řeší reálnou potřebu přehledné a efektivní
evidence zásob v~prostředí, kde se pracuje s~tisíci různých komponent.

Klíčové výstupy projektu:
\begin{itemize}
  \item Analýza 22 funkčních a 8 nefunkčních požadavků.
  \item Navžena třívrstvá architektura (React + FastAPI + relační DB).
  \item Datový model s~11 hlavními entitami pokrývající kompletní lifecycle
        součástky v~skladu.
  \item 6 UML/PlantUML diagramů dokumentujících use cases, konceptuální model, ER, sekvenční a aktivitní procesy, komponentovou a deployment architekturu.
  \item REST API s~více než 15 endpointy.
\end{itemize}

\section{Další rozvoj}

Plánovaná rozšíření pro budoucí verze systému:

\begin{description}
  \item[v1.1] Mobilní aplikace pro skenování QR kódů při příjmu a výdeji.
  \item[v1.2] Integrace s~online obchody (TME, Farnell, Mouser) pro automatické
              doplnění technických parametrů a cen.
  \item[v1.3] BOM (Bill of Materials) management -- propojení WMS s~návrhem
              DPS pro automatické ověření dostupnosti součástek.
  \item[v2.0] Multi-sklady a multi-uživatelský přístup s~pokročilými právy.
              Notifikace přes e-mail a Teams/Slack.
\end{description}

\section{Diagram tříd -- rozšiřitelnost}

Navržená architektura je připravena na budoucí rozšíření díky použití
abstraktních bázových tříd a dependency injection v~backendu, což umožní
přidávat nové typy pohybů a reportů bez zásahu do existujících modulů.
