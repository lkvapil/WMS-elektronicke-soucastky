\chapter{Datový model}
\label{chap:datovy_model}

Datový model je zachycen ve třech typech UML diagramů tříd, přičemž každý
sleduje jiný účel:

\begin{enumerate}
  \item \textbf{Konceptuální model} -- platformově nezávislý, zachycuje doménové
        koncepty, generalizaci, vazební třídy a kvalifikovanou vazbu
        (\autoref{fig:conceptual}).
  \item \textbf{Relační model} -- konsolidace pro PostgreSQL: tabulky se stereotypy
        \texttt{<<PK>>}/\texttt{<<FK>>}, trigery, mezilehlé tabulky místo M:N
        (\autoref{fig:relational}).
  \item \textbf{Grafový model} -- konsolidace pro Neo4j: uzly, pojmenované hrany
        \textsc{CAPS}, atributy na hranách jako asociační třídy (\autoref{fig:graph}).
\end{enumerate}

\section{Přehled entit}

Systém pracuje s~následujícími hlavními entitami:

\begin{description}
  \item[\texttt{Component}] Elektronická součástka -- základní katalogová informace.
  \item[\texttt{ComponentCategory}] Kategorie součástek (rezistor, kondenzátor, \ldots).
  \item[\texttt{Location}] Fyzické místo v~skladovém prostoru.
  \item[\texttt{Stock}] Zásoba -- propojení součástky, místa a množství.
  \item[\texttt{StockMovement}] Pohyb zásob (příjem, výdej, přeskladnění, korekce).
  \item[\texttt{Supplier}] Dodavatel součástek.
  \item[\texttt{Order}] Nákupní objednávka u~dodavatele.
  \item[\texttt{OrderItem}] Položka objednávky.
  \item[\texttt{Project}] Projekt, na který jsou vydávány součástky.
  \item[\texttt{User}] Uživatel systému.
  \item[\texttt{AuditLog}] Auditní záznam všech změn.
\end{description}

\section{Konceptuální model tříd}

Konceptuální model zachycuje:
\begin{itemize}
  \item \textbf{Generalizaci} -- dvouúrovňová hierarchie: \texttt{Komponenta}
        $\to$ \texttt{PasivniKomponenta}, \texttt{AktivniKomponenta}, \texttt{Konektor}
        $\to$ \texttt{Rezistor}, \texttt{Kondenzator}, \texttt{Induktor}, \ldots
  \item \textbf{Vazební třídy} -- \texttt{Zasoba} (mezi \texttt{Komponenta}
        a \texttt{SkladoveMisto}) a \texttt{PolozkaObjednavky} (mezi
        \texttt{Objednavka} a \texttt{Komponenta}).
  \item \textbf{Kvalifikovanou vazbu} -- \texttt{Dodavatel[katalogCislo]~$\to$~0..1~Komponenta}:
        pro daného dodavatele a jeho katalogové číslo existuje nejvýše jedna
        součástka. Bez kvalifikátoru by byla násobnost $0..*$.
  \item \textbf{Kompozici} -- \texttt{InventurniCyklus} je složen
        z \texttt{PolozkaInventury} (existenční závislost).
\end{itemize}

\begin{figure}[H]
  \centering
  \includegraphics[width=\textwidth]{02a_conceptual_model.png}
  \caption{Konceptuální model tříd}
  \label{fig:conceptual}
\end{figure}

\section{Relační model (PostgreSQL)}

Model pro relační databázi implementuje dědičnost pomocí vzoru
\emph{Class Table Inheritance}: každá třída v hierarchii má vlastní tabulku,
jejíž primární klíč je zároveň cizím klíčem na nadřazenou tabulku. Vztahy
M:N jsou rozloženy na dvě vazby 1:N přes mezilehlou tabulku.

\begin{figure}[H]
  \centering
  \includegraphics[width=\textwidth]{02b_relational_model.png}
  \caption{Konsolidovaný model pro relační databázi}
  \label{fig:relational}
\end{figure}

\section{Grafový model (Neo4j)}

Grafový model využívá vícenásobné labely pro dědičnost
(např. uzel s labely \texttt{:Komponenta:PasivniKomponenta:Rezistor}).
Hrany jsou pojmenovány \textsc{velkými písmeny} a mohou nést vlastní atributy
-- ty jsou zachyceny jako asociační třídy.

\begin{figure}[H]
  \centering
  \includegraphics[width=\textwidth]{02c_graph_model.png}
  \caption{Konsolidovaný model pro grafovou databázi}
  \label{fig:graph}
\end{figure}

\section{ER diagram}

Kompletní ER diagram datového modelu je zobrazen na \autoref{fig:er_diagram}.

\begin{figure}[H]
  \centering
  \includegraphics[width=\textwidth]{03_er_diagram.png}
  \caption{ER diagram datového modelu}
  \label{fig:er_diagram}
\end{figure}

\section{Popis tabulek}

\subsection{Tabulka \texttt{component\_categories}}

\begin{table}[H]
  \centering
  \caption{Tabulka component\_categories}
  \begin{tabularx}{\textwidth}{llX}
    \toprule
    \textbf{Sloupec} & \textbf{Typ}        & \textbf{Popis} \\
    \midrule
    id               & SERIAL PK           & Primární klíč \\
    name             & VARCHAR(100) UNIQUE  & Název kategorie (Rezistor, IC, \ldots) \\
    description      & TEXT                & Popis kategorie \\
    parent\_id       & INTEGER FK          & Nadřazená kategorie (stromová struktura) \\
    icon             & VARCHAR(50)         & Název ikony pro UI \\
    created\_at      & TIMESTAMPTZ         & Čas vytvoření \\
    \bottomrule
  \end{tabularx}
\end{table}

\subsection{Tabulka \texttt{components}}

\begin{table}[H]
  \centering
  \caption{Tabulka components}
  \begin{tabularx}{\textwidth}{llX}
    \toprule
    \textbf{Sloupec}    & \textbf{Typ}         & \textbf{Popis} \\
    \midrule
    id                  & SERIAL PK            & Primární klíč \\
    part\_number        & VARCHAR(100) UNIQUE  & Katalogové číslo výrobce \\
    name                & VARCHAR(200)         & Název součástky \\
    category\_id        & INTEGER FK           & Kategorie \\
    manufacturer        & VARCHAR(150)         & Výrobce \\
    value               & VARCHAR(50)          & Hodnota (10k, 100nF, \ldots) \\
    unit                & VARCHAR(20)          & Jednotka (Ω, F, H, V, A, \ldots) \\
    package             & VARCHAR(50)          & Pouzdro (SMD 0402, DIP-8, TO-92, \ldots) \\
    tolerance           & VARCHAR(20)          & Tolerance (±5 \%, ±1 \%, \ldots) \\
    voltage\_rating     & VARCHAR(30)          & Maximální napětí \\
    datasheet\_url      & TEXT                 & URL na datasheet \\
    barcode             & VARCHAR(100)         & Čárový kód EAN/QR \\
    min\_stock          & INTEGER DEFAULT 0    & Minimální zásoba \\
    description         & TEXT                 & Popis \\
    notes               & TEXT                 & Interní poznámky \\
    is\_active          & BOOLEAN DEFAULT TRUE & Aktivní / archivována \\
    created\_at         & TIMESTAMPTZ          & Čas vytvoření \\
    updated\_at         & TIMESTAMPTZ          & Čas poslední změny \\
    \bottomrule
  \end{tabularx}
\end{table}

\subsection{Tabulka \texttt{locations}}

\begin{table}[H]
  \centering
  \caption{Tabulka locations}
  \begin{tabularx}{\textwidth}{llX}
    \toprule
    \textbf{Sloupec} & \textbf{Typ}         & \textbf{Popis} \\
    \midrule
    id               & SERIAL PK            & Primární klíč \\
    code             & VARCHAR(50) UNIQUE   & Unikátní kód místa (A-01-03) \\
    name             & VARCHAR(150)         & Popis místa (Regál A, přihrádka 3) \\
    section          & VARCHAR(50)          & Sekce skladu \\
    shelf            & VARCHAR(50)          & Regál \\
    bin              & VARCHAR(50)          & Přihrádka / zásobník \\
    parent\_id       & INTEGER FK           & Nadřazené místo \\
    capacity         & INTEGER              & Kapacita (počet zásobníků) \\
    is\_active       & BOOLEAN DEFAULT TRUE & Aktivní \\
    \bottomrule
  \end{tabularx}
\end{table}

\subsection{Tabulka \texttt{stock}}

\begin{table}[H]
  \centering
  \caption{Tabulka stock (zásoby)}
  \begin{tabularx}{\textwidth}{llX}
    \toprule
    \textbf{Sloupec}  & \textbf{Typ}     & \textbf{Popis} \\
    \midrule
    id                & SERIAL PK        & Primární klíč \\
    component\_id     & INTEGER FK       & Odkaz na součástku \\
    location\_id      & INTEGER FK       & Odkaz na místo \\
    quantity          & INTEGER NOT NULL & Aktuální množství \\
    reserved\_qty     & INTEGER DEFAULT 0& Rezervované množství \\
    unit\_price       & NUMERIC(12,4)    & Pořizovací cena za kus \\
    updated\_at       & TIMESTAMPTZ      & Čas poslední aktualizace \\
    \midrule
    \multicolumn{3}{l}{\textit{UNIQUE (component\_id, location\_id)}} \\
    \bottomrule
  \end{tabularx}
\end{table}

\subsection{Tabulka \texttt{stock\_movements}}

\begin{table}[H]
  \centering
  \caption{Tabulka stock\_movements (pohyby zásob)}
  \begin{tabularx}{\textwidth}{llX}
    \toprule
    \textbf{Sloupec}    & \textbf{Typ}       & \textbf{Popis} \\
    \midrule
    id                  & SERIAL PK          & Primární klíč \\
    movement\_type      & movement\_type\_enum & RECEIPT / ISSUE / TRANSFER / ADJUSTMENT \\
    component\_id       & INTEGER FK         & Součástka \\
    from\_location\_id  & INTEGER FK         & Zdrojové místo (NULL pro RECEIPT) \\
    to\_location\_id    & INTEGER FK         & Cílové místo (NULL pro ISSUE) \\
    quantity            & INTEGER NOT NULL   & Množství pohybu \\
    unit\_price         & NUMERIC(12,4)      & Cena za kus \\
    reference           & VARCHAR(100)       & Číslo dokladu (faktura, projekt, \ldots) \\
    project\_id         & INTEGER FK         & Projekt (pro ISSUE) \\
    order\_item\_id     & INTEGER FK         & Položka objednávky (pro RECEIPT) \\
    notes               & TEXT               & Poznámka \\
    created\_by         & INTEGER FK         & Uživatel \\
    created\_at         & TIMESTAMPTZ        & Čas pohybu \\
    \bottomrule
  \end{tabularx}
\end{table}


