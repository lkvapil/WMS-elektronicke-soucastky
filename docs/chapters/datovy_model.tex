\chapter{Datový model}
\label{chap:datovy_model}

Datový model je zachycen platformově nezávislým konceptuálním diagramem tříd,
který zachycuje doménové koncepty, generalizaci a kvalifikovanou vazbu.

\section{Přehled entit}

Systém pracuje s~následujícími hlavními třídami:

\begin{description}
  \item[\texttt{PohybSkladu}] Abstraktní základní třída pro všechny pohyby skladu.
        Obsahuje společné atributy \texttt{id}, \texttt{datum} a \texttt{mnozstvi}.
  \item[\texttt{Naskladneni}] Konkrétní podtřída -- příjem zboží do skladu.
        Rozšiřuje \texttt{PohybSkladu} o~atribut \texttt{lokace}.
  \item[\texttt{Vyskladneni}] Konkrétní podtřída -- výdej zboží ze skladu.
        Rozšiřuje \texttt{PohybSkladu} o~atribut \texttt{cil}.
  \item[\texttt{Presun}] Konkrétní podtřída -- přesun zboží mezi lokacemi.
        Rozšiřuje \texttt{PohybSkladu} o~atributy \texttt{zdrojLokace}
        a \texttt{cilovaLokace}.
  \item[\texttt{SkladovaLokace}] Fyzické místo v~skladovém prostoru
        identifikované kódem \texttt{kodLokace}.
  \item[\texttt{Soucastka}] Elektronická součástka s~katalogovými údaji
        (\texttt{partNumber}, \texttt{nazev}, \texttt{kategorie}) a aktuálním množstvím.
  \item[\texttt{Dodavatel}] Dodavatel součástek s~kontaktními údaji
        (\texttt{nazev}, \texttt{zeme}, \texttt{adresa}, \texttt{email}).
\end{description}

\section{Konceptuální model tříd}

Konceptuální model zachycuje:
\begin{itemize}
  \item \textbf{Generalizaci (dědičnost)} -- abstraktní třída \texttt{PohybSkladu}
        je rodičovskou třídou pro \texttt{Naskladneni}, \texttt{Vyskladneni}
        a \texttt{Presun}. Sdílené atributy (\texttt{id}, \texttt{datum},
        \texttt{mnozstvi}) a operace (\texttt{validovatMnozstvi()},
        \texttt{ulozitZaznam()}) jsou definovány jednou v~rodičovské třídě.
  \item \textbf{Kvalifikovanou vazbu} --
        \texttt{Dodavatel[katalogCislo]~$\to$~0..1~Soucastka}:
        pro daného dodavatele a jeho konkrétní katalogové číslo existuje
        nejvýše jedna součástka. Bez kvalifikátoru by byla násobnost $0..*$.
  \item \textbf{Agregaci} -- \texttt{SkladovaLokace} agreguje $0..*$
        instancí \texttt{Soucastka} (součástka může existovat nezávisle
        na lokaci).
  \item \textbf{Asociace} -- každý pohyb skladu (\texttt{PohybSkladu})
        se vztahuje k~právě jedné \texttt{SkladovaLokace} a právě jedné
        \texttt{Soucastka}.
\end{itemize}

\begin{figure}[H]
  \centering
  \includegraphics[width=\textwidth]{02a_conceptual_model.png}
  \caption{Konceptuální model tříd}
  \label{fig:conceptual}
\end{figure}

\section{ER diagram}

Kompletní ER diagram datového modelu je zobrazen na \autoref{fig:er_diagram}.

\begin{figure}[H]
  \centering
  \includegraphics[width=\textwidth]{03_er_diagram.png}
  \caption{ER diagram datového modelu}
  \label{fig:er_diagram}
\end{figure}

\section{Popis atributů tříd}

\subsection{Třída \texttt{PohybSkladu} (abstraktní)}

\begin{table}[H]
  \centering
  \caption{Atributy a metody třídy PohybSkladu}
  \begin{tabularx}{\textwidth}{llX}
    \toprule
    \textbf{Název}        & \textbf{Typ}  & \textbf{Popis} \\
    \midrule
    id                    & int           & Jednoznačný identifikátor pohybu \\
    datum                 & DateTime      & Datum a čas provedení pohybu \\
    mnozstvi              & int           & Počet kusů daného pohybu \\
    validovatMnozstvi()   & void          & Ověří, že množství je kladné číslo \\
    ulozitZaznam()        & void          & Persistuje pohyb do úložiště \\
    \bottomrule
  \end{tabularx}
\end{table}

\subsection{Třída \texttt{Soucastka}}

\begin{table}[H]
  \centering
  \caption{Atributy a metody třídy Soucastka}
  \begin{tabularx}{\textwidth}{llX}
    \toprule
    \textbf{Název}        & \textbf{Typ}  & \textbf{Popis} \\
    \midrule
    partNumber            & String        & Katalogové číslo výrobce (unikátní) \\
    nazev                 & String        & Obchodní název součástky \\
    kategorie             & String        & Kategorie (rezistor, kondenzátor, \ldots) \\
    mnozstvi              & int           & Aktuální stav zásob \\
    aktualizovatUdaje()   & void          & Aktualizuje katalogové informace \\
    ziskatStav()          & String        & Vrátí stav zásoby (dostatek / minimum / chybí) \\
    \bottomrule
  \end{tabularx}
\end{table}

\subsection{Třída \texttt{SkladovaLokace}}

\begin{table}[H]
  \centering
  \caption{Atributy a metody třídy SkladovaLokace}
  \begin{tabularx}{\textwidth}{llX}
    \toprule
    \textbf{Název}        & \textbf{Typ} & \textbf{Popis} \\
    \midrule
    kodLokace             & String       & Unikátní kód místa v~skladu (např.\,A-01-03) \\
    popis                 & String       & Textový popis lokace \\
    pridatSoucastku()     & void         & Přiřadí součástku do lokace \\
    odebratSoucastku()    & void         & Odebere součástku z~lokace \\
    zobrazObsah()         & List         & Vrátí seznam součástek v~lokaci \\
    \bottomrule
  \end{tabularx}
\end{table}

\subsection{Třída \texttt{Dodavatel}}

\begin{table}[H]
  \centering
  \caption{Atributy a metody třídy Dodavatel}
  \begin{tabularx}{\textwidth}{llX}
    \toprule
    \textbf{Název}         & \textbf{Typ} & \textbf{Popis} \\
    \midrule
    nazev                  & String       & Obchodní název dodavatele \\
    zeme                   & String       & Země sídla dodavatele \\
    adresa                 & String       & Poštovní adresa \\
    email                  & String       & Kontaktní e-mail \\
    aktualizovatKontakt()  & void         & Aktualizuje kontaktní údaje \\
    zobrazSoucastky()      & List         & Vrátí seznam dodávaných součástek \\
    \bottomrule
  \end{tabularx}
\end{table}
