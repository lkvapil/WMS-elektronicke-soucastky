\chapter{Analýza požadavků}
\label{chap:pozadavky}

\section{Funkční požadavky}

\subsection{Správa součástek}
\begin{itemize}
  \item[\textbf{FR-01}] Systém umožní evidovat součástku s~atributy: název,
        typ, výrobce, hodnota, tolerance, pouzdro (package), popis, datasheet URL.
  \item[\textbf{FR-02}] Systém podporuje kategorie součástek (rezistory,
        kondenzátory, tranzistory, diody, IC, konektory, \ldots).
  \item[\textbf{FR-03}] Každá součástka může mít přiřazený čárový kód (EAN/QR).
  \item[\textbf{FR-04}] Uživatel může vyhledávat součástky dle libovolné kombinace parametrů.
\end{itemize}

\subsection{Skladová místa}
\begin{itemize}
  \item[\textbf{FR-05}] Systém spravuje hierarchii skladových míst:
        \emph{Sklad} $\to$ \emph{Sekce} $\to$ \emph{Regál} $\to$ \emph{Přihrádka}.
  \item[\textbf{FR-06}] Jedno skladové místo může obsahovat více druhů součástek;
        jedna součástka může být na více místech.
\end{itemize}

\subsection{Příjem a výdej}
\begin{itemize}
  \item[\textbf{FR-07}] Příjem součástek ze skladu generuje pohyb typu \texttt{PŘÍJEM}
        a aktualizuje zásoby.
  \item[\textbf{FR-08}] Výdej součástek na projekt generuje pohyb typu \texttt{VÝDEJ}
        a sníží zásoby.
  \item[\textbf{FR-09}] Přeskladnění přesouvá zásoby mezi místy bez změny celkového
        množství.
  \item[\textbf{FR-10}] Systém hlídá minimální zásoby a upozorní na podkročení limitu.
\end{itemize}

\subsection{Objednávky a dodavatelé}
\begin{itemize}
  \item[\textbf{FR-11}] Systém eviduje dodavatele s~kontaktními údaji.
  \item[\textbf{FR-12}] Uživatel může vytvořit nákupní objednávku u~dodavatele.
  \item[\textbf{FR-13}] Objednávka prochází stavy:
        \texttt{NÁVRH} $\to$ \texttt{ODESLÁNO} $\to$ \texttt{POTVRZENO} $\to$ \texttt{PŘIJATO} $\to$ \texttt{STORNOVÁNO}.
\end{itemize}

\subsection{Inventura}
\begin{itemize}
  \item[\textbf{FR-14}] Systém umožní zahájit inventurní cyklus, během nějž
        uživatel ověřuje fyzické zásoby.
  \item[\textbf{FR-15}] Systém generuje inventurní list (PDF/CSV) se seznamem
        součástek a očekávaných množství.
  \item[\textbf{FR-16}] Po ukončení inventury systém provede korekci zásob.
\end{itemize}

\subsection{Reporty}
\begin{itemize}
  \item[\textbf{FR-17}] Přehled aktuálních zásob s~filtrací dle kategorie/umístění.
  \item[\textbf{FR-18}] Historie pohybů za zvolené časové období.
  \item[\textbf{FR-19}] Součástky pod minimálním limitem zásob.
  \item[\textbf{FR-20}] Ocenění skladu (celková hodnota zásob).
\end{itemize}

\subsection{Správa uživatelů}
\begin{itemize}
  \item[\textbf{FR-21}] Systém spravuje uživatele s~rolemi: \texttt{ADMIN}, \texttt{SKLADNÍK}, \texttt{ČTENÁŘ}.
  \item[\textbf{FR-22}] Každá akce je logována s~časovým razítkem a uživatelem.
\end{itemize}

\section{Nefunkční požadavky}

\begin{table}[H]
  \centering
  \caption{Nefunkční požadavky systému}
  \label{tab:nfr}
  \begin{tabularx}{\textwidth}{llX}
    \toprule
    \textbf{ID}   & \textbf{Kategorie}   & \textbf{Požadavek} \\
    \midrule
    NFR-01 & Výkon         & Vyhledávání vrátí výsledky do 500 ms pro zásoby do 100\,000 položek. \\
    NFR-02 & Dostupnost    & Systém musí být dostupný 99,5 \% času (plánovaná údržba vyjmuta). \\
    NFR-03 & Bezpečnost    & Hesla uložena jako bcrypt hash; komunikace pouze přes HTTPS. \\
    NFR-04 & Škálovatelnost& Architektura umožní horizontální škálování backendu. \\
    NFR-05 & Přenositelnost& Nasazení pomocí Docker Compose; nezávislé na OS. \\
    NFR-06 & Použitelnost  & Webové rozhraní přístupné bez speciálního SW v~moderním prohlížeči. \\
    NFR-07 & Lokalizace    & Primární jazyk čeština; systém připraven na vícejazyčnost (i18n). \\
    NFR-08 & Auditovatelnost& Každá změna stavu skladu musí být dohledatelná v~auditním logu. \\
    \bottomrule
  \end{tabularx}
\end{table}

\section{Případy užití -- přehled}

Hlavní aktéři systému:
\begin{itemize}
  \item \textbf{Skladník} -- provádí příjem, výdej, přeskladnění a inventuru.
  \item \textbf{Administrátor} -- spravuje uživatele, nastavení, kategorie a místa.
  \item \textbf{Čtenář} -- prohlíží zásoby a reporty (pouze čtení).
  \item \textbf{Systém} -- automatické upozornění na nízké zásoby, plánované reporty.
\end{itemize}

Podrobný Use Case diagram je zobrazen v~\autoref{fig:use_case}.

\begin{figure}[H]
  \centering
  \includegraphics[width=\textwidth]{01_use_case.png}
  \caption{Use Case diagram systému WMS -- elektronické součástky}
  \label{fig:use_case}
\end{figure}
