\chapter{Úvod}
\label{chap:uvod}

\section{Motivace}

Správa skladu elektronických součástek představuje náročný logistický úkol.
Moderní elektronika využívá tisíce různých komponent -- od běžných rezistorů
a kondenzátorů přes specializované integrované obvody až po konektory a
mechanické díly. Bez kvalitního informačního systému dochází k:

\begin{itemize}
  \item duplicitním nákupům téhož zboží,
  \item ztrátě přehledu o aktuálních zásobách,
  \item zbytečnému zdržování při hledání součástek v~neoznačených zásobnících,
  \item potížím při plánování výroby a prototypování,
  \item nesnadné dohledatelnosti dodavatelů a nákupních cen.
\end{itemize}

Systém \wms{} -- Elektronické součástky si klade za cíl tyto problémy
eliminovat a přinést přehlednou, efektivní a rozšiřitelnou správu skladových
zásob.

\section{Cíle projektu}

\begin{enumerate}
  \item Vytvořit centrální evidenci všech elektronických součástek včetně
        jejich technických parametrů.
  \item Zavést přesné sledování fyzických umístění v~skladovém prostoru
        (regály, přihrádky, zásobníky).
  \item Automatizovat procesy příjmu a výdeje zboží s~vazbou na projekty.
  \item Umožnit rychlé vyhledávání součástek dle parametrů (hodnota, pouzdro,
        výrobce, kategorie).
  \item Generovat reporty o stavu skladu a pohybech zásob.
  \item Spravovat dodavatele a objednávky.
  \item Upozorňovat na součástky pod minimální hranicí zásob.
\end{enumerate}

\section{Rozsah dokumentu}

Tento dokument pokrývá:
\begin{itemize}
  \item analýzu funkčních a nefunkčních požadavků (\autoref{chap:pozadavky}),
  \item popis navrhované architektury systému (\autoref{chap:architektura}),
  \item datový model a ER diagram (\autoref{chap:datovy_model}),
  \item popis klíčových procesů formou UML diagramů (\autoref{chap:procesy}).
\end{itemize}

\section{Použité nástroje a technologie}

\begin{table}[H]
  \centering
  \caption{Přehled použitých nástrojů}
  \label{tab:nastroje}
  \begin{tabularx}{\textwidth}{llX}
    \toprule
    \textbf{Oblast}       & \textbf{Nástroj / Technologie} & \textbf{Popis} \\
    \midrule
    Dokumentace           & \LaTeX, Biber                  & Sazba odborného textu \\
    Diagramy              & PlantUML                       & UML a ER diagramy \\
    Databáze              & PostgreSQL 15                  & Relační databáze \\
    Backend               & Python / FastAPI               & REST API server \\
    Frontend              & React + TypeScript             & Webová aplikace \\
    Kontejnerizace        & Docker, Docker Compose         & Nasazení prostředí \\
    Správa kódu           & Git, GitHub                    & Verzování zdrojového kódu \\
    \bottomrule
  \end{tabularx}
\end{table}
