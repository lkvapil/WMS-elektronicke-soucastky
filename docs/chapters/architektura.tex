\chapter{Architektura systému}
\label{chap:architektura}

\section{Přehled architektury}

Systém je navržen jako třívrstvá webová aplikace s~jasně oddělenými vrstvami:

\begin{description}
  \item[Prezentační vrstva] Webová aplikace napsaná v~React + TypeScript, která
        komunikuje s~backendem prostřednictvím REST API.
  \item[Aplikační vrstva] Python backend s~frameworkem FastAPI implementující
        business logiku, autentizaci (JWT) a validaci dat.
  \item[Datová vrstva] Relační databáze správující persistenci dat;
        přístup přes ORM SQLAlchemy.
\end{description}

Komponentový diagram systému je zobrazen v~\autoref{fig:component}.

\begin{figure}[H]
  \centering
  \includegraphics[width=\textwidth]{07_component_diagram.png}
  \caption{Komponentový diagram systému}
  \label{fig:component}
\end{figure}

\section{REST API -- přehled endpointů}

\begin{table}[H]
  \centering
  \caption{Klíčové REST API endpointy}
  \label{tab:api}
  \begin{tabularx}{\textwidth}{llX}
    \toprule
    \textbf{Metoda} & \textbf{Endpoint}                & \textbf{Popis} \\
    \midrule
    GET    & \texttt{/api/components}          & Seznam součástek (s filtrací) \\
    POST   & \texttt{/api/components}          & Nová součástka \\
    GET    & \texttt{/api/components/\{id\}}   & Detail součástky \\
    PUT    & \texttt{/api/components/\{id\}}   & Aktualizace součástky \\
    DELETE & \texttt{/api/components/\{id\}}   & Smazání součástky \\
    \midrule
    GET    & \texttt{/api/stock}               & Aktuální zásoby \\
    POST   & \texttt{/api/stock/receive}       & Příjem na sklad \\
    POST   & \texttt{/api/stock/issue}         & Výdej ze skladu \\
    POST   & \texttt{/api/stock/transfer}      & Přeskladnění \\
    \midrule
    GET    & \texttt{/api/locations}           & Skladová místa \\
    GET    & \texttt{/api/suppliers}           & Dodavatelé \\
    GET    & \texttt{/api/orders}              & Objednávky \\
    POST   & \texttt{/api/orders}              & Nová objednávka \\
    PATCH  & \texttt{/api/orders/\{id\}/status}& Změna stavu objednávky \\
    \midrule
    GET    & \texttt{/api/reports/stock}       & Report stavu skladu \\
    GET    & \texttt{/api/reports/movements}   & Report pohybů \\
    GET    & \texttt{/api/reports/low-stock}   & Součástky pod limitem \\
    \bottomrule
  \end{tabularx}
\end{table}

\section{Deployment diagram}

Systém je provozován pomocí Docker Compose. Deployment diagram je na
\autoref{fig:deployment}.

\begin{figure}[H]
  \centering
  \includegraphics[width=0.85\textwidth]{08_deployment_diagram.png}
  \caption{Deployment diagram -- nasazení systému pomocí Docker}
  \label{fig:deployment}
\end{figure}

\section{Bezpečnostní model}

Autentizace je realizována pomocí JWT tokenů (JSON Web Token):
\begin{enumerate}
  \item Uživatel odešle přihlašovací údaje na \texttt{POST /api/auth/login}.
  \item Server ověří heslo (bcrypt), vygeneruje access token (platnost 15 min)
        a refresh token (platnost 7 dní).
  \item Klient přikládá access token v~hlavičce \texttt{Authorization: Bearer \{token\}}.
  \item Po vypršení access tokenu si klient vyžádá nový pomocí refresh tokenu.
\end{enumerate}

Oprávnění jsou řešena pomocí RBAC (Role-Based Access Control):

\begin{table}[H]
  \centering
  \caption{Matice oprávnění dle rolí}
  \label{tab:rbac}
  \begin{tabular}{lccc}
    \toprule
    \textbf{Operace}      & \textbf{Čtenář} & \textbf{Skladník} & \textbf{Admin} \\
    \midrule
    Prohlížení součástek  & \checkmark & \checkmark & \checkmark \\
    Editace součástek     & ---        & \checkmark & \checkmark \\
    Příjem / Výdej        & ---        & \checkmark & \checkmark \\
    Inventura             & ---        & \checkmark & \checkmark \\
    Správa dodavatelů     & ---        & \checkmark & \checkmark \\
    Správa uživatelů      & ---        & ---        & \checkmark \\
    Konfigurace systému   & ---        & ---        & \checkmark \\
    \bottomrule
  \end{tabular}
\end{table}
