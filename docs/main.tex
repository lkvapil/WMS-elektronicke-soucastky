\documentclass[12pt, a4paper]{report}

%% ── Kódování a jazyk ────────────────────────────────────────────────────────
\usepackage[utf8]{inputenc}
\usepackage[T1]{fontenc}
\usepackage[czech]{babel}

%% ── Typografie ──────────────────────────────────────────────────────────────
\usepackage{lmodern}
\usepackage{microtype}
\usepackage{setspace}
\onehalfspacing

%% ── Geometrie stránky ───────────────────────────────────────────────────────
\usepackage[
  top=25mm, bottom=25mm,
  left=35mm, right=20mm,
  headheight=15pt
]{geometry}

%% ── Záhlaví a zápatí ────────────────────────────────────────────────────────
\usepackage{fancyhdr}
\pagestyle{fancy}
\fancyhf{}
\fancyhead[L]{\leftmark}
\fancyhead[R]{\thepage}
\renewcommand{\headrulewidth}{0.4pt}

%% ── Grafika ─────────────────────────────────────────────────────────────────
\usepackage{graphicx}
\graphicspath{{./images/}}
\usepackage{float}
\usepackage{caption}
\usepackage{subcaption}

%% ── Barvy a zvýraznění ──────────────────────────────────────────────────────
\usepackage[dvipsnames]{xcolor}
\definecolor{wmsblue}{RGB}{0, 83, 156}
\definecolor{wmsgray}{RGB}{100, 100, 100}
\definecolor{wmsgreen}{RGB}{34, 139, 34}

%% ── Tabulky ─────────────────────────────────────────────────────────────────
\usepackage{booktabs}
\usepackage{longtable}
\usepackage{array}
\usepackage{tabularx}

%% ── Výpisy kódu ─────────────────────────────────────────────────────────────
\usepackage{listings}
\lstset{
  basicstyle=\ttfamily\small,
  breaklines=true,
  frame=single,
  backgroundcolor=\color{gray!10},
  keywordstyle=\color{wmsblue}\bfseries,
  commentstyle=\color{wmsgray}\itshape,
  stringstyle=\color{wmsgreen},
  numbers=left,
  numberstyle=\tiny\color{wmsgray},
  numbersep=5pt,
  extendedchars=true,
  literate=
    {á}{{\'{a}}}1 {é}{{\'{e}}}1 {í}{{\'{\i}}}1 {ó}{{\'{o}}}1 {ú}{{\'{u}}}1
    {ě}{{\v{e}}}1 {š}{{\v{s}}}1 {č}{{\v{c}}}1 {ř}{{\v{r}}}1 {ž}{{\v{z}}}1
    {ů}{{\r{u}}}1 {Á}{{\'{A}}}1 {É}{{\'{E}}}1 {Í}{{\'{I}}}1 {Ó}{{\'{O}}}1
    {Ú}{{\'{U}}}1 {Ě}{{\v{E}}}1 {Š}{{\v{S}}}1 {Č}{{\v{C}}}1 {Ř}{{\v{R}}}1
    {Ž}{{\v{Z}}}1 {Ů}{{\r{U}}}1,
}

%% ── Matematika ───────────────────────────────────────────────────────────────
\usepackage{amsmath}
\usepackage{amssymb}
\usepackage{mathtools}

%% ── Hypertextové odkazy ──────────────────────────────────────────────────────
\usepackage[
  colorlinks=true,
  linkcolor=black,
  citecolor=black,
  urlcolor=black,
  pdftitle={WMS – Elektronické součástky},
  pdfauthor={Bc. Tomáš Simandl, Bc. Tomáš Szetei, Bc. Lukáš Kvapil},
  pdfsubject={Warehouse Management System pro elektronické součástky}
]{hyperref}

%% ── Bibliografie ─────────────────────────────────────────────────────────────
\usepackage[backend=biber, style=numeric, sorting=nyt]{biblatex}
\addbibresource{literatura.bib}

%% ── Tikz (volitelně pro jednoduché grafy inline) ─────────────────────────────
\usepackage{tikz}
\usetikzlibrary{shapes, arrows.meta, positioning, fit, calc}

%% ── Vlastní příkazy ──────────────────────────────────────────────────────────
\newcommand{\wms}{\textsc{WMS}}
\newcommand{\todo}[1]{\textcolor{red}{\textbf{TODO:} #1}}

%% ─────────────────────────────────────────────────────────────────────────────
%%  DOKUMENT
%% ─────────────────────────────────────────────────────────────────────────────
\begin{document}

%% ── Titulní strana ───────────────────────────────────────────────────────────
\begin{titlepage}
  \centering
  \vspace*{2cm}

  {\rule{\linewidth}{1.5pt}}
  \vspace{0.5cm}

  {\Huge\bfseries
    Systém pro správu skladu\\[0.4em]
    elektronických součástek
  }

  \vspace{0.3cm}
  {\rule{\linewidth}{1.5pt}}

  \vspace{1.5cm}

  {\LARGE\bfseries WMS -- Elektronické součástky}

  \vspace{0.8cm}
  {\large Projektová dokumentace}

  \vfill

  \begin{tabular}{ll}
    \textbf{Autoři:}   & Bc.~Tomáš Simandl  \\[4pt]
                       & Bc.~Tomáš Szetei    \\[4pt]
                       & Bc.~Lukáš Kvapil   \\[4pt]
    \textbf{Datum:}    & \today                \\[4pt]
  \end{tabular}

  \vspace{1cm}
  {\small Dokumentace systému pro skladovou evidenci elektronických součástek}
\end{titlepage}

%% ── Abstakt ──────────────────────────────────────────────────────────────────
\chapter*{Abstrakt}
\addcontentsline{toc}{chapter}{Abstrakt}

Tento dokument popisuje návrh a implementaci systému pro správu skladu
(\textit{Warehouse Management System}, \wms) určeného pro evidenci elektronických
součástek. Systém umožňuje sledovat zásoby součástek (rezistory, kondenzátory,
tranzistory, integrované obvody a další), spravovat skladová místa, zpracovávat
příjem a výdej zboží, generovat objednávky u dodavatelů a tvořit reporty o stavu
skladu.

Dokument obsahuje analýzu požadavků, datový model, architekturu systému a
UML/PlantUML diagramy popisující klíčové procesy.

\vspace{1cm}
\noindent\textbf{Klíčová slova:} WMS, sklad, elektronické součástky, správa
zásob, inventura, příjem, výdej, objednávky.

%% ── Obsah ────────────────────────────────────────────────────────────────────
\tableofcontents
\listoffigures
\listoftables

%% ── Kapitoly ─────────────────────────────────────────────────────────────────
\chapter{Úvod}
\label{chap:uvod}

\section{Motivace}

Správa skladu elektronických součástek představuje náročný logistický úkol.
Moderní elektronika využívá tisíce různých komponent -- od běžných rezistorů
a kondenzátorů přes specializované integrované obvody až po konektory a
mechanické díly. Bez kvalitního informačního systému dochází k:

\begin{itemize}
  \item duplicitním nákupům téhož zboží,
  \item ztrátě přehledu o aktuálních zásobách,
  \item zbytečnému zdržování při hledání součástek v~neoznačených zásobnících,
  \item potížím při plánování výroby a prototypování,
  \item nesnadné dohledatelnosti dodavatelů a nákupních cen.
\end{itemize}

Systém \wms{} -- Elektronické součástky si klade za cíl tyto problémy
eliminovat a přinést přehlednou, efektivní a rozšiřitelnou správu skladových
zásob.

\section{Cíle projektu}

\begin{enumerate}
  \item Vytvořit centrální evidenci všech elektronických součástek včetně
        jejich technických parametrů.
  \item Zavést přesné sledování fyzických umístění v~skladovém prostoru
        (regály, přihrádky, zásobníky).
  \item Automatizovat procesy příjmu a výdeje zboží s~vazbou na projekty.
  \item Umožnit rychlé vyhledávání součástek dle parametrů (hodnota, pouzdro,
        výrobce, kategorie).
  \item Generovat reporty o stavu skladu a pohybech zásob.
  \item Spravovat dodavatele a objednávky.
  \item Upozorňovat na součástky pod minimální hranicí zásob.
\end{enumerate}

\section{Rozsah dokumentu}

Tento dokument pokrývá:
\begin{itemize}
  \item analýzu funkčních a nefunkčních požadavků (\autoref{chap:pozadavky}),
  \item popis navrhované architektury systému (\autoref{chap:architektura}),
  \item datový model a ER diagram (\autoref{chap:datovy_model}),
  \item popis klíčových procesů formou UML diagramů (\autoref{chap:procesy}).
\end{itemize}

\section{Použité nástroje a technologie}

\begin{table}[H]
  \centering
  \caption{Přehled použitých nástrojů}
  \label{tab:nastroje}
  \begin{tabularx}{\textwidth}{llX}
    \toprule
    \textbf{Oblast}       & \textbf{Nástroj / Technologie} & \textbf{Popis} \\
    \midrule
    Dokumentace           & \LaTeX, Biber                  & Sazba odborného textu \\
    Diagramy              & PlantUML                       & UML a ER diagramy \\
    Databáze              & PostgreSQL 15                  & Relační databáze \\
    Backend               & Python / FastAPI               & REST API server \\
    Frontend              & React + TypeScript             & Webová aplikace \\
    Kontejnerizace        & Docker, Docker Compose         & Nasazení prostředí \\
    Správa kódu           & Git, GitHub                    & Verzování zdrojového kódu \\
    \bottomrule
  \end{tabularx}
\end{table}

\chapter{Analýza požadavků}
\label{chap:pozadavky}

\section{Funkční požadavky}

\subsection{Správa součástek}
\begin{itemize}
  \item[\textbf{FR-01}] Systém umožní evidovat součástku s~atributy: název,
        typ, výrobce, hodnota, tolerance, pouzdro (package), popis, datasheet URL.
  \item[\textbf{FR-02}] Systém podporuje kategorie součástek (rezistory,
        kondenzátory, tranzistory, diody, IC, konektory, \ldots).
  \item[\textbf{FR-03}] Každá součástka může mít přiřazený čárový kód (EAN/QR).
  \item[\textbf{FR-04}] Uživatel může vyhledávat součástky dle libovolné kombinace parametrů.
\end{itemize}

\subsection{Skladová místa}
\begin{itemize}
  \item[\textbf{FR-05}] Systém spravuje hierarchii skladových míst:
        \emph{Sklad} $\to$ \emph{Sekce} $\to$ \emph{Regál} $\to$ \emph{Přihrádka}.
  \item[\textbf{FR-06}] Jedno skladové místo může obsahovat více druhů součástek;
        jedna součástka může být na více místech.
\end{itemize}

\subsection{Příjem a výdej}
\begin{itemize}
  \item[\textbf{FR-07}] Příjem součástek ze skladu generuje pohyb typu \texttt{PŘÍJEM}
        a aktualizuje zásoby.
  \item[\textbf{FR-08}] Výdej součástek na projekt generuje pohyb typu \texttt{VÝDEJ}
        a sníží zásoby.
  \item[\textbf{FR-09}] Přeskladnění přesouvá zásoby mezi místy bez změny celkového
        množství.
  \item[\textbf{FR-10}] Systém hlídá minimální zásoby a upozorní na podkročení limitu.
\end{itemize}

\subsection{Objednávky a dodavatelé}
\begin{itemize}
  \item[\textbf{FR-11}] Systém eviduje dodavatele s~kontaktními údaji.
  \item[\textbf{FR-12}] Uživatel může vytvořit nákupní objednávku u~dodavatele.
  \item[\textbf{FR-13}] Objednávka prochází stavy:
        \texttt{NÁVRH} $\to$ \texttt{ODESLÁNO} $\to$ \texttt{POTVRZENO} $\to$ \texttt{PŘIJATO} $\to$ \texttt{STORNOVÁNO}.
\end{itemize}

\subsection{Inventura}
\begin{itemize}
  \item[\textbf{FR-14}] Systém umožní zahájit inventurní cyklus, během nějž
        uživatel ověřuje fyzické zásoby.
  \item[\textbf{FR-15}] Systém generuje inventurní list (PDF/CSV) se seznamem
        součástek a očekávaných množství.
  \item[\textbf{FR-16}] Po ukončení inventury systém provede korekci zásob.
\end{itemize}

\subsection{Reporty}
\begin{itemize}
  \item[\textbf{FR-17}] Přehled aktuálních zásob s~filtrací dle kategorie/umístění.
  \item[\textbf{FR-18}] Historie pohybů za zvolené časové období.
  \item[\textbf{FR-19}] Součástky pod minimálním limitem zásob.
  \item[\textbf{FR-20}] Ocenění skladu (celková hodnota zásob).
\end{itemize}

\subsection{Správa uživatelů}
\begin{itemize}
  \item[\textbf{FR-21}] Systém spravuje uživatele s~rolemi: \texttt{ADMIN}, \texttt{SKLADNÍK}, \texttt{ČTENÁŘ}.
  \item[\textbf{FR-22}] Každá akce je logována s~časovým razítkem a uživatelem.
\end{itemize}

\section{Nefunkční požadavky}

\begin{table}[H]
  \centering
  \caption{Nefunkční požadavky systému}
  \label{tab:nfr}
  \begin{tabularx}{\textwidth}{llX}
    \toprule
    \textbf{ID}   & \textbf{Kategorie}   & \textbf{Požadavek} \\
    \midrule
    NFR-01 & Výkon         & Vyhledávání vrátí výsledky do 500 ms pro zásoby do 100\,000 položek. \\
    NFR-02 & Dostupnost    & Systém musí být dostupný 99,5 \% času (plánovaná údržba vyjmuta). \\
    NFR-03 & Bezpečnost    & Hesla uložena jako bcrypt hash; komunikace pouze přes HTTPS. \\
    NFR-04 & Škálovatelnost& Architektura umožní horizontální škálování backendu. \\
    NFR-05 & Přenositelnost& Nasazení pomocí Docker Compose; nezávislé na OS. \\
    NFR-06 & Použitelnost  & Webové rozhraní přístupné bez speciálního SW v~moderním prohlížeči. \\
    NFR-07 & Lokalizace    & Primární jazyk čeština; systém připraven na vícejazyčnost (i18n). \\
    NFR-08 & Auditovatelnost& Každá změna stavu skladu musí být dohledatelná v~auditním logu. \\
    \bottomrule
  \end{tabularx}
\end{table}

\section{Případy užití -- přehled}

Hlavní aktéři systému:
\begin{itemize}
  \item \textbf{Skladník} -- provádí příjem, výdej, přeskladnění a inventuru.
  \item \textbf{Administrátor} -- spravuje uživatele, nastavení, kategorie a místa.
  \item \textbf{Čtenář} -- prohlíží zásoby a reporty (pouze čtení).
  \item \textbf{Systém} -- automatické upozornění na nízké zásoby, plánované reporty.
\end{itemize}

Podrobný Use Case diagram je zobrazen v~\autoref{fig:use_case}.

\begin{figure}[H]
  \centering
  \includegraphics[width=\textwidth]{01_use_case.png}
  \caption{Use Case diagram systému WMS -- elektronické součástky}
  \label{fig:use_case}
\end{figure}

\chapter{Architektura systému}
\label{chap:architektura}

\section{Přehled architektury}

Systém je navržen jako třívrstvá webová aplikace s~jasně oddělenými vrstvami:

\begin{description}
  \item[Prezentační vrstva] Webová aplikace napsaná v~React + TypeScript, která
        komunikuje s~backendem prostřednictvím REST API.
  \item[Aplikační vrstva] Python backend s~frameworkem FastAPI implementující
        business logiku, autentizaci (JWT) a validaci dat.
  \item[Datová vrstva] Relační databáze správující persistenci dat;
        přístup přes ORM SQLAlchemy.
\end{description}

Komponentový diagram systému je zobrazen v~\autoref{fig:component}.

\begin{figure}[H]
  \centering
  \includegraphics[width=\textwidth]{07_component_diagram.png}
  \caption{Komponentový diagram systému}
  \label{fig:component}
\end{figure}

\section{REST API -- přehled endpointů}

\begin{table}[H]
  \centering
  \caption{Klíčové REST API endpointy}
  \label{tab:api}
  \begin{tabularx}{\textwidth}{llX}
    \toprule
    \textbf{Metoda} & \textbf{Endpoint}                & \textbf{Popis} \\
    \midrule
    GET    & \texttt{/api/components}          & Seznam součástek (s filtrací) \\
    POST   & \texttt{/api/components}          & Nová součástka \\
    GET    & \texttt{/api/components/\{id\}}   & Detail součástky \\
    PUT    & \texttt{/api/components/\{id\}}   & Aktualizace součástky \\
    DELETE & \texttt{/api/components/\{id\}}   & Smazání součástky \\
    \midrule
    GET    & \texttt{/api/stock}               & Aktuální zásoby \\
    POST   & \texttt{/api/stock/receive}       & Příjem na sklad \\
    POST   & \texttt{/api/stock/issue}         & Výdej ze skladu \\
    POST   & \texttt{/api/stock/transfer}      & Přeskladnění \\
    \midrule
    GET    & \texttt{/api/locations}           & Skladová místa \\
    GET    & \texttt{/api/suppliers}           & Dodavatelé \\
    GET    & \texttt{/api/orders}              & Objednávky \\
    POST   & \texttt{/api/orders}              & Nová objednávka \\
    PATCH  & \texttt{/api/orders/\{id\}/status}& Změna stavu objednávky \\
    \midrule
    GET    & \texttt{/api/reports/stock}       & Report stavu skladu \\
    GET    & \texttt{/api/reports/movements}   & Report pohybů \\
    GET    & \texttt{/api/reports/low-stock}   & Součástky pod limitem \\
    \bottomrule
  \end{tabularx}
\end{table}

\section{Deployment diagram}

Systém je provozován pomocí Docker Compose. Deployment diagram je na
\autoref{fig:deployment}.

\begin{figure}[H]
  \centering
  \includegraphics[width=0.85\textwidth]{08_deployment_diagram.png}
  \caption{Deployment diagram -- nasazení systému pomocí Docker}
  \label{fig:deployment}
\end{figure}

\section{Bezpečnostní model}

Autentizace je realizována pomocí JWT tokenů (JSON Web Token):
\begin{enumerate}
  \item Uživatel odešle přihlašovací údaje na \texttt{POST /api/auth/login}.
  \item Server ověří heslo (bcrypt), vygeneruje access token (platnost 15 min)
        a refresh token (platnost 7 dní).
  \item Klient přikládá access token v~hlavičce \texttt{Authorization: Bearer \{token\}}.
  \item Po vypršení access tokenu si klient vyžádá nový pomocí refresh tokenu.
\end{enumerate}

Oprávnění jsou řešena pomocí RBAC (Role-Based Access Control):

\begin{table}[H]
  \centering
  \caption{Matice oprávnění dle rolí}
  \label{tab:rbac}
  \begin{tabular}{lccc}
    \toprule
    \textbf{Operace}      & \textbf{Čtenář} & \textbf{Skladník} & \textbf{Admin} \\
    \midrule
    Prohlížení součástek  & \checkmark & \checkmark & \checkmark \\
    Editace součástek     & ---        & \checkmark & \checkmark \\
    Příjem / Výdej        & ---        & \checkmark & \checkmark \\
    Inventura             & ---        & \checkmark & \checkmark \\
    Správa dodavatelů     & ---        & \checkmark & \checkmark \\
    Správa uživatelů      & ---        & ---        & \checkmark \\
    Konfigurace systému   & ---        & ---        & \checkmark \\
    \bottomrule
  \end{tabular}
\end{table}

\chapter{Datový model}
\label{chap:datovy_model}

Datový model je zachycen ve třech typech UML diagramů tříd, přičemž každý
sleduje jiný účel:

\begin{enumerate}
  \item \textbf{Konceptuální model} -- platformově nezávislý, zachycuje doménové
        koncepty, generalizaci, vazební třídy a kvalifikovanou vazbu
        (\autoref{fig:conceptual}).
  \item \textbf{Relační model} -- konsolidace pro PostgreSQL: tabulky se stereotypy
        \texttt{<<PK>>}/\texttt{<<FK>>}, trigery, mezilehlé tabulky místo M:N
        (\autoref{fig:relational}).
  \item \textbf{Grafový model} -- konsolidace pro Neo4j: uzly, pojmenované hrany
        \textsc{CAPS}, atributy na hranách jako asociační třídy (\autoref{fig:graph}).
\end{enumerate}

\section{Přehled entit}

Systém pracuje s~následujícími hlavními entitami:

\begin{description}
  \item[\texttt{Component}] Elektronická součástka -- základní katalogová informace.
  \item[\texttt{ComponentCategory}] Kategorie součástek (rezistor, kondenzátor, \ldots).
  \item[\texttt{Location}] Fyzické místo v~skladovém prostoru.
  \item[\texttt{Stock}] Zásoba -- propojení součástky, místa a množství.
  \item[\texttt{StockMovement}] Pohyb zásob (příjem, výdej, přeskladnění, korekce).
  \item[\texttt{Supplier}] Dodavatel součástek.
  \item[\texttt{Order}] Nákupní objednávka u~dodavatele.
  \item[\texttt{OrderItem}] Položka objednávky.
  \item[\texttt{Project}] Projekt, na který jsou vydávány součástky.
  \item[\texttt{User}] Uživatel systému.
  \item[\texttt{AuditLog}] Auditní záznam všech změn.
\end{description}

\section{Konceptuální model tříd}

Konceptuální model zachycuje:
\begin{itemize}
  \item \textbf{Generalizaci} -- dvouúrovňová hierarchie: \texttt{Komponenta}
        $\to$ \texttt{PasivniKomponenta}, \texttt{AktivniKomponenta}, \texttt{Konektor}
        $\to$ \texttt{Rezistor}, \texttt{Kondenzator}, \texttt{Induktor}, \ldots
  \item \textbf{Vazební třídy} -- \texttt{Zasoba} (mezi \texttt{Komponenta}
        a \texttt{SkladoveMisto}) a \texttt{PolozkaObjednavky} (mezi
        \texttt{Objednavka} a \texttt{Komponenta}).
  \item \textbf{Kvalifikovanou vazbu} -- \texttt{Dodavatel[katalogCislo]~$\to$~0..1~Komponenta}:
        pro daného dodavatele a jeho katalogové číslo existuje nejvýše jedna
        součástka. Bez kvalifikátoru by byla násobnost $0..*$.
  \item \textbf{Kompozici} -- \texttt{InventurniCyklus} je složen
        z \texttt{PolozkaInventury} (existenční závislost).
\end{itemize}

\begin{figure}[H]
  \centering
  \includegraphics[width=\textwidth]{02a_conceptual_model.png}
  \caption{Konceptuální model tříd}
  \label{fig:conceptual}
\end{figure}

\section{Relační model (PostgreSQL)}

Model pro relační databázi implementuje dědičnost pomocí vzoru
\emph{Class Table Inheritance}: každá třída v hierarchii má vlastní tabulku,
jejíž primární klíč je zároveň cizím klíčem na nadřazenou tabulku. Vztahy
M:N jsou rozloženy na dvě vazby 1:N přes mezilehlou tabulku.

\begin{figure}[H]
  \centering
  \includegraphics[width=\textwidth]{02b_relational_model.png}
  \caption{Konsolidovaný model pro relační databázi}
  \label{fig:relational}
\end{figure}

\section{Grafový model (Neo4j)}

Grafový model využívá vícenásobné labely pro dědičnost
(např. uzel s labely \texttt{:Komponenta:PasivniKomponenta:Rezistor}).
Hrany jsou pojmenovány \textsc{velkými písmeny} a mohou nést vlastní atributy
-- ty jsou zachyceny jako asociační třídy.

\begin{figure}[H]
  \centering
  \includegraphics[width=\textwidth]{02c_graph_model.png}
  \caption{Konsolidovaný model pro grafovou databázi}
  \label{fig:graph}
\end{figure}

\section{ER diagram}

Kompletní ER diagram datového modelu je zobrazen na \autoref{fig:er_diagram}.

\begin{figure}[H]
  \centering
  \includegraphics[width=\textwidth]{03_er_diagram.png}
  \caption{ER diagram datového modelu}
  \label{fig:er_diagram}
\end{figure}

\section{Popis tabulek}

\subsection{Tabulka \texttt{component\_categories}}

\begin{table}[H]
  \centering
  \caption{Tabulka component\_categories}
  \begin{tabularx}{\textwidth}{llX}
    \toprule
    \textbf{Sloupec} & \textbf{Typ}        & \textbf{Popis} \\
    \midrule
    id               & SERIAL PK           & Primární klíč \\
    name             & VARCHAR(100) UNIQUE  & Název kategorie (Rezistor, IC, \ldots) \\
    description      & TEXT                & Popis kategorie \\
    parent\_id       & INTEGER FK          & Nadřazená kategorie (stromová struktura) \\
    icon             & VARCHAR(50)         & Název ikony pro UI \\
    created\_at      & TIMESTAMPTZ         & Čas vytvoření \\
    \bottomrule
  \end{tabularx}
\end{table}

\subsection{Tabulka \texttt{components}}

\begin{table}[H]
  \centering
  \caption{Tabulka components}
  \begin{tabularx}{\textwidth}{llX}
    \toprule
    \textbf{Sloupec}    & \textbf{Typ}         & \textbf{Popis} \\
    \midrule
    id                  & SERIAL PK            & Primární klíč \\
    part\_number        & VARCHAR(100) UNIQUE  & Katalogové číslo výrobce \\
    name                & VARCHAR(200)         & Název součástky \\
    category\_id        & INTEGER FK           & Kategorie \\
    manufacturer        & VARCHAR(150)         & Výrobce \\
    value               & VARCHAR(50)          & Hodnota (10k, 100nF, \ldots) \\
    unit                & VARCHAR(20)          & Jednotka (Ω, F, H, V, A, \ldots) \\
    package             & VARCHAR(50)          & Pouzdro (SMD 0402, DIP-8, TO-92, \ldots) \\
    tolerance           & VARCHAR(20)          & Tolerance (±5 \%, ±1 \%, \ldots) \\
    voltage\_rating     & VARCHAR(30)          & Maximální napětí \\
    datasheet\_url      & TEXT                 & URL na datasheet \\
    barcode             & VARCHAR(100)         & Čárový kód EAN/QR \\
    min\_stock          & INTEGER DEFAULT 0    & Minimální zásoba \\
    description         & TEXT                 & Popis \\
    notes               & TEXT                 & Interní poznámky \\
    is\_active          & BOOLEAN DEFAULT TRUE & Aktivní / archivována \\
    created\_at         & TIMESTAMPTZ          & Čas vytvoření \\
    updated\_at         & TIMESTAMPTZ          & Čas poslední změny \\
    \bottomrule
  \end{tabularx}
\end{table}

\subsection{Tabulka \texttt{locations}}

\begin{table}[H]
  \centering
  \caption{Tabulka locations}
  \begin{tabularx}{\textwidth}{llX}
    \toprule
    \textbf{Sloupec} & \textbf{Typ}         & \textbf{Popis} \\
    \midrule
    id               & SERIAL PK            & Primární klíč \\
    code             & VARCHAR(50) UNIQUE   & Unikátní kód místa (A-01-03) \\
    name             & VARCHAR(150)         & Popis místa (Regál A, přihrádka 3) \\
    section          & VARCHAR(50)          & Sekce skladu \\
    shelf            & VARCHAR(50)          & Regál \\
    bin              & VARCHAR(50)          & Přihrádka / zásobník \\
    parent\_id       & INTEGER FK           & Nadřazené místo \\
    capacity         & INTEGER              & Kapacita (počet zásobníků) \\
    is\_active       & BOOLEAN DEFAULT TRUE & Aktivní \\
    \bottomrule
  \end{tabularx}
\end{table}

\subsection{Tabulka \texttt{stock}}

\begin{table}[H]
  \centering
  \caption{Tabulka stock (zásoby)}
  \begin{tabularx}{\textwidth}{llX}
    \toprule
    \textbf{Sloupec}  & \textbf{Typ}     & \textbf{Popis} \\
    \midrule
    id                & SERIAL PK        & Primární klíč \\
    component\_id     & INTEGER FK       & Odkaz na součástku \\
    location\_id      & INTEGER FK       & Odkaz na místo \\
    quantity          & INTEGER NOT NULL & Aktuální množství \\
    reserved\_qty     & INTEGER DEFAULT 0& Rezervované množství \\
    unit\_price       & NUMERIC(12,4)    & Pořizovací cena za kus \\
    updated\_at       & TIMESTAMPTZ      & Čas poslední aktualizace \\
    \midrule
    \multicolumn{3}{l}{\textit{UNIQUE (component\_id, location\_id)}} \\
    \bottomrule
  \end{tabularx}
\end{table}

\subsection{Tabulka \texttt{stock\_movements}}

\begin{table}[H]
  \centering
  \caption{Tabulka stock\_movements (pohyby zásob)}
  \begin{tabularx}{\textwidth}{llX}
    \toprule
    \textbf{Sloupec}    & \textbf{Typ}       & \textbf{Popis} \\
    \midrule
    id                  & SERIAL PK          & Primární klíč \\
    movement\_type      & movement\_type\_enum & RECEIPT / ISSUE / TRANSFER / ADJUSTMENT \\
    component\_id       & INTEGER FK         & Součástka \\
    from\_location\_id  & INTEGER FK         & Zdrojové místo (NULL pro RECEIPT) \\
    to\_location\_id    & INTEGER FK         & Cílové místo (NULL pro ISSUE) \\
    quantity            & INTEGER NOT NULL   & Množství pohybu \\
    unit\_price         & NUMERIC(12,4)      & Cena za kus \\
    reference           & VARCHAR(100)       & Číslo dokladu (faktura, projekt, \ldots) \\
    project\_id         & INTEGER FK         & Projekt (pro ISSUE) \\
    order\_item\_id     & INTEGER FK         & Položka objednávky (pro RECEIPT) \\
    notes               & TEXT               & Poznámka \\
    created\_by         & INTEGER FK         & Uživatel \\
    created\_at         & TIMESTAMPTZ        & Čas pohybu \\
    \bottomrule
  \end{tabularx}
\end{table}



\chapter{Formální specifikace dotazů}
\label{chap:formalni}

Tato kapitola zavádí formální notaci pro popis dotazů nad datovým modelem
\wms{}. Notace je inspirována lambda kalkulem a množinovou algebrou
a je platformově nezávislá.

%% ── Definice notace ──────────────────────────────────────────────────────────
\section{Definice notace}

\paragraph{Universum.}
$\mathcal{U}$ označuje množinu všech uzlů (objektů) v systému. Množiny jsou
reprezentovány jako charakteristické funkce; $x \in A$ je zkrácený zápis za
$(A\ x)$.

\paragraph{Selekce (filtr).}
Infixový operátor $\mathbin{/\!/}$ filtruje množinu:
\[
  A \mathbin{/\!/} \Lambda \;\coloneqq\; \{\, x \mid (x \in A) \wedge (\Lambda\, x) \,\}
\]

\paragraph{Transformace (map).}
Infixový operátor $\gg$ mapuje množinu:
\[
  A \gg f \;\coloneqq\; \{\, f(x) \mid x \in A \,\}
\]

\paragraph{Flatten.}
Funkce $\operatorname{flatten}$ sloučí množinu množin:
\[
  \operatorname{flatten}(M) \;=\; \bigcup M
\]

\paragraph{Skládání do struktury.}
Infixový operátor $\oplus$ vytváří strukturu z hodnot:
\[
  a \oplus b \;\coloneqq\; \operatorname{Struct}(a,\, b)
\]

\paragraph{Atributy.}
Tečková notace je zkratkou pro funkční zápis:
$s.\mathit{věk} \coloneqq \mathit{věk}(s)$.

\paragraph{Orientovaná hrana.}
$x \xrightarrow{\text{Relace}} y$ znamená $(x,y) \in \text{Relace}$.

%% ── Příklady dotazů ──────────────────────────────────────────────────────────
\section{Příklady dotazů nad modelem WMS}

\subsection{Součástky pod minimálním stavem}

Množina součástek, jejichž celkový stav zásob klesl pod minimum:

\[
  K_{\min}
  \;=\;
  \text{Komponenty}
  \mathbin{/\!/}
  \bigl(\lambda k \mid k.\mathit{totalStock} < k.\mathit{minStock}\bigr)
\]

kde $k.\mathit{totalStock}$ je agregovaná hodnota:

\[
  k.\mathit{totalStock}
  \;\coloneqq\;
  \sum_{\,z\,:\; k \xrightarrow{\text{ULOZENA\_NA}} z}
    z.\mathit{quantity}
\]

Výstupní projekce (jméno, číslo, stav):

\[
  K_{\min}
  \gg
  \bigl(\lambda k \mid k.\mathit{name} \oplus k.\mathit{partNumber} \oplus k.\mathit{totalStock}\bigr)
\]

\subsection{Celková hodnota skladu}

Suma hodnot všech zásob:

\[
  \mathit{HodnotaSkladu}
  \;=\;
  \sum_{z \in \text{Zásoby}} z.\mathit{quantity} \times z.\mathit{unitPrice}
\]

\subsection{Katalog součástek dodavatele}

Všechny součástky nabízené konkrétním dodavatelem $d_0$:

\[
  \operatorname{flatten}\!\Bigl(
    \bigl\{\, d_0 \,\bigr\}
    \gg
    \bigl(\lambda d \mid
      \bigl\{\, k \mid d \xrightarrow{\text{DODAVANA\_OD}} k \bigr\}
    \bigr)
  \Bigr)
\]

\subsection{Pohyby výdeje na projekt}

Pohyby typu \texttt{VYDEJ} navázané na projekt $p_0$, provedené za
dané časové období $[t_1, t_2]$:

\[
  \text{Pohyby}
  \mathbin{/\!/}
  \Bigl(
    \lambda m \mid
      m.\mathit{movementType} = \texttt{VYDEJ}
      \;\wedge\;
      m \xrightarrow{\text{VYDANA\_NA}} p_0
      \;\wedge\;
      t_1 \leq m.\mathit{createdAt} \leq t_2
  \Bigr)
\]

Výsledná projekce (součástka, množství, datum):

\[
  \ldots \gg
  \bigl(
    \lambda m \mid m.\mathit{component} \oplus m.\mathit{quantity} \oplus m.\mathit{createdAt}
  \bigr)
\]

\subsection{Složité dotazy -- inventurní rozdíly}

Položky inventury, kde se zjištěný počet liší od evidovaného
(schválené i neschválené):

\[
  \text{PolozkyInventury}
  \mathbin{/\!/}
  \bigl(\lambda p \mid p.\mathit{countedQty} \neq p.\mathit{expectedQty}\bigr)
  \gg
  \bigl(
    \lambda p \mid
      p.\mathit{component}
      \oplus p.\mathit{expectedQty}
      \oplus p.\mathit{countedQty}
      \oplus (p.\mathit{countedQty} - p.\mathit{expectedQty})
  \bigr)
\]

\chapter{Popis procesů}
\label{chap:procesy}

\section{Proces příjmu součástek}

Příjem součástek na sklad je jedním z~nejčastějších procesů. Probíhá typicky
na základě dodávky od dodavatele, přičemž může, ale nemusí, navazovat na
předchozí objednávku.

Sekvenční diagram příjmu je zobrazen na \autoref{fig:seq_prijem}.

\begin{figure}[H]
  \centering
  \includegraphics[width=0.9\textwidth]{04_sequence_prijem.png}
  \caption{Sekvenční diagram -- Příjem součástek na sklad}
  \label{fig:seq_prijem}
\end{figure}

\subsection{Popis kroků příjmu}

\begin{enumerate}
  \item Skladník otevře formulář příjmu a naskenuje/zadá číslo dokladu nebo
        vyhledá objednávku.
  \item Pro každou přijímanou položku:
    \begin{enumerate}
      \item Vyhledá součástku (dle part number nebo skenování čárového kódu).
      \item Určí cílové skladové místo.
      \item Zadá přijímané množství a jednotkovou cenu.
    \end{enumerate}
  \item Systém ověří validitu dat (množství $> 0$, místo existuje, součástka aktivní).
  \item Systém vytvoří záznamy v~\texttt{stock\_movements} a aktualizuje tabulku
        \texttt{stock}.
  \item Pokud příjem navazuje na objednávku, systém aktualizuje stav objednávky.
  \item Systém zkontroluje, zda příjem neobnovil zásoby nad minimum (zruší upozornění).
  \item Vytiskne se příjemka.
\end{enumerate}

\section{Proces výdeje součástek}

Výdej se provádí při přidělení součástek konkrétnímu projektu nebo opravě.
Sekvenční diagram je na \autoref{fig:seq_vydej}.

\begin{figure}[H]
  \centering
  \includegraphics[width=0.9\textwidth]{05_sequence_vydej.png}
  \caption{Sekvenční diagram -- Výdej součástek ze skladu}
  \label{fig:seq_vydej}
\end{figure}

\section{Proces inventury}

Inventura je periodický proces ověření fyzického stavu skladu. Diagram
aktivit je zobrazen na \autoref{fig:activity_inventura}.

\begin{figure}[H]
  \centering
  \includegraphics[width=0.85\textwidth]{06_activity_inventura.png}
  \caption{Diagram aktivit -- Průběh inventury}
  \label{fig:activity_inventura}
\end{figure}

\subsection{Fáze inventurního procesu}

\begin{description}
  \item[Příprava] Administrátor zahájí inventurní cyklus. Systém uzamkne zásoby
        pro editaci a vygeneruje inventurní listy dle skladových míst.
  \item[Fyzické sčítání] Skladník prochází místa a zadává fyzicky zjištěné
        množství. Systém umožňuje zadávání postupně po sekcích.
  \item[Porovnání] Systém porovná evidované a fyzicky zjištěné stavy.
        Zobrazí seznam rozdílů (přebytky / manka).
  \item[Schválení a korekce] Administrátor přezkoumá rozdíly. Po schválení
        systém vytvoří pohyby typu \texttt{ADJUSTMENT} a aktualizuje zásoby.
  \item[Uzavření] Inventura je uzavřena, zásoby odemčeny, vygenerován protokol.
\end{description}

\section{Stav objednávky -- stavový diagram}

\begin{figure}[H]
  \centering
  % Jednoduchý stavový diagram pomocí TikZ (bez PlantUML obrázku)
  \begin{tikzpicture}[
    state/.style={draw, rounded corners, minimum width=2.5cm, minimum height=0.8cm,
                  fill=blue!10, font=\small},
    arrow/.style={-{Latex[length=2mm]}, thick}
  ]
    \node[state] (draft)     at (0,0)    {NÁVRH};
    \node[state] (sent)      at (4,0)    {ODESLÁNO};
    \node[state] (confirmed) at (8,0)    {POTVRZENO};
    \node[state] (received)  at (8,-2)   {PŘIJATO};
    \node[state] (cancelled) at (4,-2)   {STORNOVÁNO};

    \draw[arrow] (draft)     -- node[above,font=\tiny]{odeslat} (sent);
    \draw[arrow] (sent)      -- node[above,font=\tiny]{potvrdit} (confirmed);
    \draw[arrow] (confirmed) -- node[right,font=\tiny]{přijmout} (received);
    \draw[arrow] (sent)      -- node[right,font=\tiny]{storno} (cancelled);
    \draw[arrow] (confirmed) -- node[above,font=\tiny]{storno} (cancelled);
    \draw[arrow] (draft)     to[bend right=30] node[below,font=\tiny]{storno} (cancelled);
  \end{tikzpicture}
  \caption{Stavový diagram objednávky}
  \label{fig:stav_objednavky}
\end{figure}

\chapter{Závěr}
\label{chap:zaver}

\section{Shrnutí}

Tato dokumentace popsala návrh systému pro správu skladu elektronických
součástek (\wms{}). Systém řeší reálnou potřebu přehledné a efektivní
evidence zásob v~prostředí, kde se pracuje s~tisíci různých komponent.

Klíčové výstupy projektu:
\begin{itemize}
  \item Analýza 22 funkčních a 8 nefunkčních požadavků.
  \item Navžena třívrstvá architektura (React + FastAPI + relační DB).
  \item Datový model s~11 hlavními entitami pokrývající kompletní lifecycle
        součástky v~skladu.
  \item 6 UML/PlantUML diagramů dokumentujících use cases, konceptuální model, ER, sekvenční a aktivitní procesy, komponentovou a deployment architekturu.
  \item REST API s~více než 15 endpointy.
\end{itemize}

\section{Další rozvoj}

Plánovaná rozšíření pro budoucí verze systému:

\begin{description}
  \item[v1.1] Mobilní aplikace pro skenování QR kódů při příjmu a výdeji.
  \item[v1.2] Integrace s~online obchody (TME, Farnell, Mouser) pro automatické
              doplnění technických parametrů a cen.
  \item[v1.3] BOM (Bill of Materials) management -- propojení WMS s~návrhem
              DPS pro automatické ověření dostupnosti součástek.
  \item[v2.0] Multi-sklady a multi-uživatelský přístup s~pokročilými právy.
              Notifikace přes e-mail a Teams/Slack.
\end{description}

\section{Diagram tříd -- rozšiřitelnost}

Navržená architektura je připravena na budoucí rozšíření díky použití
abstraktních bázových tříd a dependency injection v~backendu, což umožní
přidávat nové typy pohybů a reportů bez zásahu do existujících modulů.


%% ── Bibliografie ─────────────────────────────────────────────────────────────
\printbibliography[heading=bibintoc, title={Literatura a zdroje}]

%% ── Přílohy ──────────────────────────────────────────────────────────────────
\appendix
\chapter{Slovník pojmů a zkratek}

\begin{description}
  \item[WMS] Warehouse Management System -- Systém pro správu skladu
  \item[EAN] European Article Number -- čárový kód
  \item[QR] Quick Response -- 2D čárový kód
  \item[IC] Integrated Circuit -- integrovaný obvod
  \item[SMD] Surface-Mount Device -- povrchová montáž
  \item[THT] Through-Hole Technology -- průchozí montáž
  \item[REST] Representational State Transfer -- architektonický styl API
  \item[CRUD] Create, Read, Update, Delete -- základní databázové operace
  \item[ER] Entity-Relationship -- diagram datových entit a vztahů
  \item[UML] Unified Modeling Language -- unifikovaný modelovací jazyk
\end{description}

\chapter{Instalace a konfigurace}
\label{app:instalace}

\section*{Požadavky na prostředí}
\begin{itemize}
  \item Python 3.11+, Node.js 20+ nebo Java 17+
  \item PostgreSQL 15+
  \item Docker (volitelně)
  \item PlantUML JAR pro regeneraci diagramů
\end{itemize}

\section*{Generování diagramů z PlantUML}
\begin{lstlisting}[language=bash, caption={Generovani PNG diagramu z PUML souboru}]
# Instalace PlantUML (macOS)
brew install plantuml

# Generování vsech diagramu najednou
plantuml -tpng diagrams/*.puml -o ../docs/images/

# Nebo pomoci Makefile
make diagrams
\end{lstlisting}

\end{document}
